% Modèle logique de données - E-commerce Comparable

\documentclass[a4paper,12pt]{article}
\usepackage[utf8]{inputenc}
\usepackage[T1]{fontenc}
\usepackage[french]{babel}
\usepackage{geometry}
\usepackage{tikz}
\usetikzlibrary{positioning,shapes,arrows.meta}
\geometry{margin=2cm}
\title{Modèle logique de données\newline\large E-commerce Comparable}
\author{Nestor}
\date{Août 2025}

\begin{document}
\maketitle

\section*{Diagramme logique}

\begin{center}
\begin{tikzpicture}[
  entity/.style={rectangle, draw, rounded corners, fill=blue!10, minimum width=3.5cm, minimum height=1.2cm},
  rel/.style={->, thick},
  node distance=2.5cm and 4cm
]
% Entities
\node[entity] (users) {\textbf{users}\newline id (PK)\newline name\newline email\newline password\newline role\newline created\_at};
\node[entity, below=of users] (orders) {\textbf{orders}\newline id (PK)\newline user\_id (FK)\newline total\_price\newline status\newline created\_at};
\node[entity, right=of users] (categories) {\textbf{categories}\newline id (PK)\newline name};
\node[entity, below=of categories] (products) {\textbf{products}\newline id (PK)\newline title\newline description\newline price\newline offer\newline category\_id (FK)\newline image\newline status\newline created\_at};
\node[entity, below=of orders] (order_items) {\textbf{order\_items}\newline id (PK)\newline order\_id (FK)\newline product\_id (FK)\newline quantity\newline price};
% Relations
\draw[rel] (users) -- node[right]{places} (orders);
\draw[rel] (orders) -- node[right]{contains} (order_items);
\draw[rel] (products) -- node[left]{is in} (order_items);
\draw[rel] (categories) -- node[right]{has} (products);
\end{tikzpicture}
\end{center}

\section*{Détail des tables}
\begin{itemize}
  \item \textbf{users} : id (PK), name, email, password, role, created\_at
  \item \textbf{products} : id (PK), title, description, price, offer, category\_id (FK), image, status, created\_at
  \item \textbf{categories} : id (PK), name
  \item \textbf{orders} : id (PK), user\_id (FK), total\_price, status, created\_at
  \item \textbf{order\_items} : id (PK), order\_id (FK), product\_id (FK), quantity, price
\end{itemize}

\section*{Contraintes et relations}
\begin{itemize}
  \item Un utilisateur peut passer plusieurs commandes.
  \item Une commande contient plusieurs articles.
  \item Un article de commande référence un produit.
  \item Un produit appartient à une catégorie.
\end{itemize}

\end{document}
